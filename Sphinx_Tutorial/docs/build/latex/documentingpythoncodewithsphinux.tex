%% Generated by Sphinx.
\def\sphinxdocclass{report}
\documentclass[letterpaper,10pt,english]{sphinxmanual}
\ifdefined\pdfpxdimen
   \let\sphinxpxdimen\pdfpxdimen\else\newdimen\sphinxpxdimen
\fi \sphinxpxdimen=.75bp\relax
\ifdefined\pdfimageresolution
    \pdfimageresolution= \numexpr \dimexpr1in\relax/\sphinxpxdimen\relax
\fi
%% let collapsible pdf bookmarks panel have high depth per default
\PassOptionsToPackage{bookmarksdepth=5}{hyperref}

\PassOptionsToPackage{booktabs}{sphinx}
\PassOptionsToPackage{colorrows}{sphinx}

\PassOptionsToPackage{warn}{textcomp}
\usepackage[utf8]{inputenc}
\ifdefined\DeclareUnicodeCharacter
% support both utf8 and utf8x syntaxes
  \ifdefined\DeclareUnicodeCharacterAsOptional
    \def\sphinxDUC#1{\DeclareUnicodeCharacter{"#1}}
  \else
    \let\sphinxDUC\DeclareUnicodeCharacter
  \fi
  \sphinxDUC{00A0}{\nobreakspace}
  \sphinxDUC{2500}{\sphinxunichar{2500}}
  \sphinxDUC{2502}{\sphinxunichar{2502}}
  \sphinxDUC{2514}{\sphinxunichar{2514}}
  \sphinxDUC{251C}{\sphinxunichar{251C}}
  \sphinxDUC{2572}{\textbackslash}
\fi
\usepackage{cmap}
\usepackage[T1]{fontenc}
\usepackage{amsmath,amssymb,amstext}
\usepackage{babel}



\usepackage{tgtermes}
\usepackage{tgheros}
\renewcommand{\ttdefault}{txtt}



\usepackage[Sonny]{fncychap}
\ChNameVar{\Large\normalfont\sffamily}
\ChTitleVar{\Large\normalfont\sffamily}
\usepackage{sphinx}

\fvset{fontsize=auto}
\usepackage{geometry}


% Include hyperref last.
\usepackage{hyperref}
% Fix anchor placement for figures with captions.
\usepackage{hypcap}% it must be loaded after hyperref.
% Set up styles of URL: it should be placed after hyperref.
\urlstyle{same}

\addto\captionsenglish{\renewcommand{\contentsname}{Contents:}}

\usepackage{sphinxmessages}
\setcounter{tocdepth}{1}



\title{Documenting Python Code with Sphinux}
\date{Jan 28, 2025}
\release{\textquotesingle{}0.0.1\textquotesingle{}}
\author{Baijaynta Bhattacharyya}
\newcommand{\sphinxlogo}{\vbox{}}
\renewcommand{\releasename}{Release}
\makeindex
\begin{document}

\ifdefined\shorthandoff
  \ifnum\catcode`\=\string=\active\shorthandoff{=}\fi
  \ifnum\catcode`\"=\active\shorthandoff{"}\fi
\fi

\pagestyle{empty}
\sphinxmaketitle
\pagestyle{plain}
\sphinxtableofcontents
\pagestyle{normal}
\phantomsection\label{\detokenize{index::doc}}


\sphinxAtStartPar
Add your content using \sphinxcode{\sphinxupquote{reStructuredText}} syntax. See the
\sphinxhref{https://www.sphinx-doc.org/en/master/usage/restructuredtext/index.html}{reStructuredText}
documentation for details.

\sphinxstepscope


\chapter{Sphinx\_Tutorial}
\label{\detokenize{modules:sphinx-tutorial}}\label{\detokenize{modules::doc}}
\sphinxstepscope


\section{add module}
\label{\detokenize{add:module-add}}\label{\detokenize{add:add-module}}\label{\detokenize{add::doc}}\index{module@\spxentry{module}!add@\spxentry{add}}\index{add@\spxentry{add}!module@\spxentry{module}}\index{add\_numbers() (in module add)@\spxentry{add\_numbers()}\spxextra{in module add}}

\begin{fulllineitems}
\phantomsection\label{\detokenize{add:add.add_numbers}}
\pysigstartsignatures
\pysiglinewithargsret{\sphinxcode{\sphinxupquote{add.}}\sphinxbfcode{\sphinxupquote{add\_numbers}}}{\sphinxparam{\DUrole{n}{a}}\sphinxparamcomma \sphinxparam{\DUrole{n}{b}}}{}
\pysigstopsignatures
\sphinxAtStartPar
Add two numbers and return the result.
\begin{quote}\begin{description}
\sphinxlineitem{Parameters}\begin{itemize}
\item {} 
\sphinxAtStartPar
\sphinxstyleliteralstrong{\sphinxupquote{a}} (\sphinxstyleliteralemphasis{\sphinxupquote{int}}\sphinxstyleliteralemphasis{\sphinxupquote{ or }}\sphinxstyleliteralemphasis{\sphinxupquote{float}}) \sphinxhyphen{}\sphinxhyphen{} The first number.

\item {} 
\sphinxAtStartPar
\sphinxstyleliteralstrong{\sphinxupquote{b}} (\sphinxstyleliteralemphasis{\sphinxupquote{int}}\sphinxstyleliteralemphasis{\sphinxupquote{ or }}\sphinxstyleliteralemphasis{\sphinxupquote{float}}) \sphinxhyphen{}\sphinxhyphen{} The second number.

\end{itemize}

\sphinxlineitem{Returns}
\sphinxAtStartPar
The sum of \sphinxtitleref{a} and \sphinxtitleref{b}.

\sphinxlineitem{Return type}
\sphinxAtStartPar
int or float

\end{description}\end{quote}
\subsubsection*{Example}

\begin{sphinxVerbatim}[commandchars=\\\{\}]
\PYG{g+gp}{\PYGZgt{}\PYGZgt{}\PYGZgt{} }\PYG{n}{add\PYGZus{}numbers}\PYG{p}{(}\PYG{l+m+mi}{2}\PYG{p}{,} \PYG{l+m+mi}{3}\PYG{p}{)}
\PYG{g+go}{5}
\end{sphinxVerbatim}

\end{fulllineitems}


\sphinxstepscope


\section{multiply module}
\label{\detokenize{multiply:module-multiply}}\label{\detokenize{multiply:multiply-module}}\label{\detokenize{multiply::doc}}\index{module@\spxentry{module}!multiply@\spxentry{multiply}}\index{multiply@\spxentry{multiply}!module@\spxentry{module}}\index{multiply\_numbers() (in module multiply)@\spxentry{multiply\_numbers()}\spxextra{in module multiply}}

\begin{fulllineitems}
\phantomsection\label{\detokenize{multiply:multiply.multiply_numbers}}
\pysigstartsignatures
\pysiglinewithargsret{\sphinxcode{\sphinxupquote{multiply.}}\sphinxbfcode{\sphinxupquote{multiply\_numbers}}}{\sphinxparam{\DUrole{n}{a}}\sphinxparamcomma \sphinxparam{\DUrole{n}{b}}}{}
\pysigstopsignatures
\sphinxAtStartPar
Multiply two numbers and return the result.
\begin{quote}\begin{description}
\sphinxlineitem{Parameters}\begin{itemize}
\item {} 
\sphinxAtStartPar
\sphinxstyleliteralstrong{\sphinxupquote{a}} (\sphinxstyleliteralemphasis{\sphinxupquote{int}}\sphinxstyleliteralemphasis{\sphinxupquote{ or }}\sphinxstyleliteralemphasis{\sphinxupquote{float}}) \sphinxhyphen{}\sphinxhyphen{} The first number.

\item {} 
\sphinxAtStartPar
\sphinxstyleliteralstrong{\sphinxupquote{b}} (\sphinxstyleliteralemphasis{\sphinxupquote{int}}\sphinxstyleliteralemphasis{\sphinxupquote{ or }}\sphinxstyleliteralemphasis{\sphinxupquote{float}}) \sphinxhyphen{}\sphinxhyphen{} The second number.

\end{itemize}

\sphinxlineitem{Returns}
\sphinxAtStartPar
The product of \sphinxtitleref{a} and \sphinxtitleref{b}.

\sphinxlineitem{Return type}
\sphinxAtStartPar
int or float

\end{description}\end{quote}
\subsubsection*{Example}

\begin{sphinxVerbatim}[commandchars=\\\{\}]
\PYG{g+gp}{\PYGZgt{}\PYGZgt{}\PYGZgt{} }\PYG{n}{multiply\PYGZus{}numbers}\PYG{p}{(}\PYG{l+m+mi}{2}\PYG{p}{,} \PYG{l+m+mi}{3}\PYG{p}{)}
\PYG{g+go}{6}
\PYG{g+gp}{\PYGZgt{}\PYGZgt{}\PYGZgt{} }\PYG{n}{multiply\PYGZus{}numbers}\PYG{p}{(}\PYG{l+m+mf}{5.2}\PYG{p}{,}\PYG{l+m+mf}{3.4}\PYG{p}{)}
\PYG{g+go}{17.68}
\end{sphinxVerbatim}

\end{fulllineitems}


\sphinxstepscope


\section{subtract module}
\label{\detokenize{subtract:module-subtract}}\label{\detokenize{subtract:subtract-module}}\label{\detokenize{subtract::doc}}\index{module@\spxentry{module}!subtract@\spxentry{subtract}}\index{subtract@\spxentry{subtract}!module@\spxentry{module}}\index{subtract\_numbers() (in module subtract)@\spxentry{subtract\_numbers()}\spxextra{in module subtract}}

\begin{fulllineitems}
\phantomsection\label{\detokenize{subtract:subtract.subtract_numbers}}
\pysigstartsignatures
\pysiglinewithargsret{\sphinxcode{\sphinxupquote{subtract.}}\sphinxbfcode{\sphinxupquote{subtract\_numbers}}}{\sphinxparam{\DUrole{n}{a}}\sphinxparamcomma \sphinxparam{\DUrole{n}{b}}}{}
\pysigstopsignatures
\sphinxAtStartPar
Subtract two numbers and return the result.
\begin{quote}\begin{description}
\sphinxlineitem{Parameters}\begin{itemize}
\item {} 
\sphinxAtStartPar
\sphinxstyleliteralstrong{\sphinxupquote{a}} (\sphinxstyleliteralemphasis{\sphinxupquote{int}}\sphinxstyleliteralemphasis{\sphinxupquote{ or }}\sphinxstyleliteralemphasis{\sphinxupquote{float}}) \sphinxhyphen{}\sphinxhyphen{} The first number.

\item {} 
\sphinxAtStartPar
\sphinxstyleliteralstrong{\sphinxupquote{b}} (\sphinxstyleliteralemphasis{\sphinxupquote{int}}\sphinxstyleliteralemphasis{\sphinxupquote{ or }}\sphinxstyleliteralemphasis{\sphinxupquote{float}}) \sphinxhyphen{}\sphinxhyphen{} The second number.

\end{itemize}

\sphinxlineitem{Returns}
\sphinxAtStartPar
The subtract of \sphinxtitleref{a} and \sphinxtitleref{b}.

\sphinxlineitem{Return type}
\sphinxAtStartPar
int or float

\end{description}\end{quote}
\subsubsection*{Example}

\begin{sphinxVerbatim}[commandchars=\\\{\}]
\PYG{g+gp}{\PYGZgt{}\PYGZgt{}\PYGZgt{} }\PYG{n}{subtract\PYGZus{}numbers}\PYG{p}{(}\PYG{l+m+mi}{2}\PYG{p}{,} \PYG{l+m+mi}{3}\PYG{p}{)}
\PYG{g+go}{\PYGZhy{}1}
\end{sphinxVerbatim}

\end{fulllineitems}



\renewcommand{\indexname}{Python Module Index}
\begin{sphinxtheindex}
\let\bigletter\sphinxstyleindexlettergroup
\bigletter{a}
\item\relax\sphinxstyleindexentry{add}\sphinxstyleindexpageref{add:\detokenize{module-add}}
\indexspace
\bigletter{m}
\item\relax\sphinxstyleindexentry{multiply}\sphinxstyleindexpageref{multiply:\detokenize{module-multiply}}
\indexspace
\bigletter{s}
\item\relax\sphinxstyleindexentry{subtract}\sphinxstyleindexpageref{subtract:\detokenize{module-subtract}}
\end{sphinxtheindex}

\renewcommand{\indexname}{Index}
\printindex
\end{document}