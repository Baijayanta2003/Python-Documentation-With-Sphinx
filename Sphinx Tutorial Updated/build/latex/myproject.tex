%% Generated by Sphinx.
\def\sphinxdocclass{report}
\documentclass[a4paper,14pt,oneside,english,openany]{sphinxmanual}
\ifdefined\pdfpxdimen
   \let\sphinxpxdimen\pdfpxdimen\else\newdimen\sphinxpxdimen
\fi \sphinxpxdimen=.75bp\relax
\ifdefined\pdfimageresolution
    \pdfimageresolution= \numexpr \dimexpr1in\relax/\sphinxpxdimen\relax
\fi
%% let collapsible pdf bookmarks panel have high depth per default
\PassOptionsToPackage{bookmarksdepth=5}{hyperref}

\PassOptionsToPackage{booktabs}{sphinx}
\PassOptionsToPackage{colorrows}{sphinx}

\PassOptionsToPackage{warn}{textcomp}
\usepackage[utf8]{inputenc}
\ifdefined\DeclareUnicodeCharacter
% support both utf8 and utf8x syntaxes
  \ifdefined\DeclareUnicodeCharacterAsOptional
    \def\sphinxDUC#1{\DeclareUnicodeCharacter{"#1}}
  \else
    \let\sphinxDUC\DeclareUnicodeCharacter
  \fi
  \sphinxDUC{00A0}{\nobreakspace}
  \sphinxDUC{2500}{\sphinxunichar{2500}}
  \sphinxDUC{2502}{\sphinxunichar{2502}}
  \sphinxDUC{2514}{\sphinxunichar{2514}}
  \sphinxDUC{251C}{\sphinxunichar{251C}}
  \sphinxDUC{2572}{\textbackslash}
\fi
\usepackage{cmap}
\usepackage[T1]{fontenc}
\usepackage{amsmath,amssymb,amstext}
\usepackage{babel}



        \usepackage{palatino}
    


\usepackage[Bjarne]{fncychap}
\usepackage{sphinx}

\fvset{fontsize=auto}
\usepackage{geometry}
\usepackage{fancyhdr}

% Include hyperref last.
\usepackage{hyperref}
% Fix anchor placement for figures with captions.
\usepackage{hypcap}% it must be loaded after hyperref.
% Set up styles of URL: it should be placed after hyperref.
\urlstyle{same}

\addto\captionsenglish{\renewcommand{\contentsname}{Contents:}}

\usepackage{sphinxmessages}
\setcounter{tocdepth}{1}


        \usepackage{palatino} 
	\usepackage{amsmath,amssymb}
	 
        \usepackage{fvextra}  % Better verbatim environments
    

\title{MyProject}
\date{Feb 01, 2025}
\release{0.0.1}
\author{MyAuthor}
\newcommand{\sphinxlogo}{\vbox{}}
\renewcommand{\releasename}{Release}
\makeindex
\begin{document}

\ifdefined\shorthandoff
  \ifnum\catcode`\=\string=\active\shorthandoff{=}\fi
  \ifnum\catcode`\"=\active\shorthandoff{"}\fi
\fi

\pagestyle{empty}
\sphinxmaketitle
\pagestyle{plain}
\sphinxtableofcontents
\pagestyle{normal}
\phantomsection\label{\detokenize{index::doc}}


\sphinxAtStartPar
Add your content using \sphinxcode{\sphinxupquote{reStructuredText}} syntax. See the
\sphinxhref{https://www.sphinx-doc.org/en/master/usage/restructuredtext/index.html}{reStructuredText}
documentation for details.

\sphinxstepscope


\chapter{Sphinx Tutorial Updated}
\label{\detokenize{modules:sphinx-tutorial-updated}}\label{\detokenize{modules::doc}}
\sphinxstepscope


\section{add module}
\label{\detokenize{add:module-add}}\label{\detokenize{add:add-module}}\label{\detokenize{add::doc}}\index{module@\spxentry{module}!add@\spxentry{add}}\index{add@\spxentry{add}!module@\spxentry{module}}

\subsection{Addition Module}
\label{\detokenize{add:addition-module}}\begin{equation*}
\begin{split}a + b\end{split}
\end{equation*}
\sphinxAtStartPar
This module provides a function to add two numbers.
\index{add() (in module add)@\spxentry{add()}\spxextra{in module add}}

\begin{fulllineitems}
\phantomsection\label{\detokenize{add:add.add}}
\pysigstartsignatures
\pysiglinewithargsret{\sphinxcode{\sphinxupquote{add.}}\sphinxbfcode{\sphinxupquote{add}}}{\sphinxparam{\DUrole{n}{a}}\sphinxparamcomma \sphinxparam{\DUrole{n}{b}}}{}
\pysigstopsignatures
\sphinxAtStartPar
Returns the sum of a and b.

\end{fulllineitems}


\sphinxstepscope


\section{newtest module}
\label{\detokenize{newtest:module-newtest}}\label{\detokenize{newtest:newtest-module}}\label{\detokenize{newtest::doc}}\index{module@\spxentry{module}!newtest@\spxentry{newtest}}\index{newtest@\spxentry{newtest}!module@\spxentry{module}}\index{area\_of\_ellipse() (in module newtest)@\spxentry{area\_of\_ellipse()}\spxextra{in module newtest}}

\begin{fulllineitems}
\phantomsection\label{\detokenize{newtest:newtest.area_of_ellipse}}
\pysigstartsignatures
\pysiglinewithargsret{\sphinxcode{\sphinxupquote{newtest.}}\sphinxbfcode{\sphinxupquote{area\_of\_ellipse}}}{\sphinxparam{\DUrole{n}{a}}\sphinxparamcomma \sphinxparam{\DUrole{n}{b}}}{}
\pysigstopsignatures
\sphinxAtStartPar
Computes the area of an ellipse:
\begin{equation*}
\begin{split}A = \pi a b\end{split}
\end{equation*}
\sphinxAtStartPar
where:
\sphinxhyphen{} \(A\) is the area,
\sphinxhyphen{} \(a\) is the semi\sphinxhyphen{}major axis,
\sphinxhyphen{} \(b\) is the semi\sphinxhyphen{}minor axis.
\begin{quote}\begin{description}
\sphinxlineitem{Parameters}\begin{itemize}
\item {} 
\sphinxAtStartPar
\sphinxstyleliteralstrong{\sphinxupquote{a}} \textendash{} The semi\sphinxhyphen{}major axis length.

\item {} 
\sphinxAtStartPar
\sphinxstyleliteralstrong{\sphinxupquote{b}} \textendash{} The semi\sphinxhyphen{}minor axis length.

\end{itemize}

\sphinxlineitem{Returns}
\sphinxAtStartPar
The area of the ellipse.

\sphinxlineitem{Return type}
\sphinxAtStartPar
float

\end{description}\end{quote}

\end{fulllineitems}

\index{fibonacci() (in module newtest)@\spxentry{fibonacci()}\spxextra{in module newtest}}

\begin{fulllineitems}
\phantomsection\label{\detokenize{newtest:newtest.fibonacci}}
\pysigstartsignatures
\pysiglinewithargsret{\sphinxcode{\sphinxupquote{newtest.}}\sphinxbfcode{\sphinxupquote{fibonacci}}}{\sphinxparam{\DUrole{n}{n}}}{}
\pysigstopsignatures
\sphinxAtStartPar
Computes the nth Fibonacci number using Binet’s formula:
\begin{equation*}
\begin{split}F_n = \frac{1}{\sqrt{5}} \left( \left( \frac{1 + \sqrt{5}}{2} \right)^n - \left( \frac{1 - \sqrt{5}}{2} \right)^n \right)\end{split}
\end{equation*}
\sphinxAtStartPar
where:
\sphinxhyphen{} \(F_n\) is the nth Fibonacci number.
\begin{quote}\begin{description}
\sphinxlineitem{Parameters}
\sphinxAtStartPar
\sphinxstyleliteralstrong{\sphinxupquote{n}} \textendash{} The index of the Fibonacci sequence.

\sphinxlineitem{Returns}
\sphinxAtStartPar
The nth Fibonacci number.

\sphinxlineitem{Return type}
\sphinxAtStartPar
int

\end{description}\end{quote}

\end{fulllineitems}

\index{logistic\_growth() (in module newtest)@\spxentry{logistic\_growth()}\spxextra{in module newtest}}

\begin{fulllineitems}
\phantomsection\label{\detokenize{newtest:newtest.logistic_growth}}
\pysigstartsignatures
\pysiglinewithargsret{\sphinxcode{\sphinxupquote{newtest.}}\sphinxbfcode{\sphinxupquote{logistic\_growth}}}{\sphinxparam{\DUrole{n}{P0}}\sphinxparamcomma \sphinxparam{\DUrole{n}{r}}\sphinxparamcomma \sphinxparam{\DUrole{n}{t}}\sphinxparamcomma \sphinxparam{\DUrole{n}{K}}}{}
\pysigstopsignatures
\sphinxAtStartPar
Models the population growth according to the logistic growth model:
\begin{equation*}
\begin{split}P(t) = \frac{K P_0}{P_0 + (K - P_0) e^{-rt}}\end{split}
\end{equation*}
\sphinxAtStartPar
where:
\sphinxhyphen{} \(P(t)\) is the population at time \(t\),
\sphinxhyphen{} \(P_0\) is the initial population,
\sphinxhyphen{} \(r\) is the growth rate,
\sphinxhyphen{} \(K\) is the carrying capacity of the environment.
\begin{quote}\begin{description}
\sphinxlineitem{Parameters}\begin{itemize}
\item {} 
\sphinxAtStartPar
\sphinxstyleliteralstrong{\sphinxupquote{P0}} \textendash{} Initial population size.

\item {} 
\sphinxAtStartPar
\sphinxstyleliteralstrong{\sphinxupquote{r}} \textendash{} Growth rate.

\item {} 
\sphinxAtStartPar
\sphinxstyleliteralstrong{\sphinxupquote{t}} \textendash{} Time at which to evaluate the population.

\item {} 
\sphinxAtStartPar
\sphinxstyleliteralstrong{\sphinxupquote{K}} \textendash{} Carrying capacity.

\end{itemize}

\sphinxlineitem{Returns}
\sphinxAtStartPar
The population at time \(t\).

\sphinxlineitem{Return type}
\sphinxAtStartPar
float

\end{description}\end{quote}

\end{fulllineitems}

\index{logistic\_map() (in module newtest)@\spxentry{logistic\_map()}\spxextra{in module newtest}}

\begin{fulllineitems}
\phantomsection\label{\detokenize{newtest:newtest.logistic_map}}
\pysigstartsignatures
\pysiglinewithargsret{\sphinxcode{\sphinxupquote{newtest.}}\sphinxbfcode{\sphinxupquote{logistic\_map}}}{\sphinxparam{\DUrole{n}{x0}}\sphinxparamcomma \sphinxparam{\DUrole{n}{r}}\sphinxparamcomma \sphinxparam{\DUrole{n}{n}}}{}
\pysigstopsignatures
\sphinxAtStartPar
Simulates the logistic map:
\begin{equation*}
\begin{split}x_{n+1} = r x_n (1 - x_n)\end{split}
\end{equation*}
\sphinxAtStartPar
where:
\sphinxhyphen{} \(x_{n+1}\) is the population at the next time step,
\sphinxhyphen{} \(x_n\) is the population at the current time step,
\sphinxhyphen{} \(r\) is the growth rate.

\sphinxAtStartPar
This is often used to model chaotic systems.
\begin{quote}\begin{description}
\sphinxlineitem{Parameters}\begin{itemize}
\item {} 
\sphinxAtStartPar
\sphinxstyleliteralstrong{\sphinxupquote{x0}} \textendash{} Initial population size.

\item {} 
\sphinxAtStartPar
\sphinxstyleliteralstrong{\sphinxupquote{r}} \textendash{} Growth rate parameter.

\item {} 
\sphinxAtStartPar
\sphinxstyleliteralstrong{\sphinxupquote{n}} \textendash{} Number of iterations.

\end{itemize}

\sphinxlineitem{Returns}
\sphinxAtStartPar
The population after n iterations.

\sphinxlineitem{Return type}
\sphinxAtStartPar
float

\end{description}\end{quote}

\end{fulllineitems}

\index{pendulum\_period() (in module newtest)@\spxentry{pendulum\_period()}\spxextra{in module newtest}}

\begin{fulllineitems}
\phantomsection\label{\detokenize{newtest:newtest.pendulum_period}}
\pysigstartsignatures
\pysiglinewithargsret{\sphinxcode{\sphinxupquote{newtest.}}\sphinxbfcode{\sphinxupquote{pendulum\_period}}}{\sphinxparam{\DUrole{n}{length}}\sphinxparamcomma \sphinxparam{\DUrole{n}{g}\DUrole{o}{=}\DUrole{default_value}{9.81}}}{}
\pysigstopsignatures
\sphinxAtStartPar
Calculates the period of a simple pendulum:
\begin{equation*}
\begin{split}T = 2 \pi \sqrt{\frac{l}{g}}\end{split}
\end{equation*}
\sphinxAtStartPar
where:
\sphinxhyphen{} \(T\) is the period,
\sphinxhyphen{} \(l\) is the length of the pendulum,
\sphinxhyphen{} \(g\) is the acceleration due to gravity (default value is 9.81 m/s\(\sp{\text{2}}\)).
\begin{quote}\begin{description}
\sphinxlineitem{Parameters}\begin{itemize}
\item {} 
\sphinxAtStartPar
\sphinxstyleliteralstrong{\sphinxupquote{length}} \textendash{} The length of the pendulum.

\item {} 
\sphinxAtStartPar
\sphinxstyleliteralstrong{\sphinxupquote{g}} \textendash{} The gravitational acceleration (default is 9.81 m/s\(\sp{\text{2}}\)).

\end{itemize}

\sphinxlineitem{Returns}
\sphinxAtStartPar
The period of the pendulum.

\sphinxlineitem{Return type}
\sphinxAtStartPar
float

\end{description}\end{quote}

\end{fulllineitems}

\index{quadratic\_formula() (in module newtest)@\spxentry{quadratic\_formula()}\spxextra{in module newtest}}

\begin{fulllineitems}
\phantomsection\label{\detokenize{newtest:newtest.quadratic_formula}}
\pysigstartsignatures
\pysiglinewithargsret{\sphinxcode{\sphinxupquote{newtest.}}\sphinxbfcode{\sphinxupquote{quadratic\_formula}}}{\sphinxparam{\DUrole{n}{a}}\sphinxparamcomma \sphinxparam{\DUrole{n}{b}}\sphinxparamcomma \sphinxparam{\DUrole{n}{c}}}{}
\pysigstopsignatures
\sphinxAtStartPar
Solves the quadratic equation:
\begin{equation*}
\begin{split}ax^2 + bx + c = 0\end{split}
\end{equation*}
\sphinxAtStartPar
Using the quadratic formula:
\begin{equation*}
\begin{split}x = \frac{-b \pm \sqrt{b^2 - 4ac}}{2a}\end{split}
\end{equation*}
\sphinxAtStartPar
where:
\sphinxhyphen{} \(a\), \(b\), and \(c\) are the coefficients of the quadratic equation.
\sphinxhyphen{} \(x\) is the solution to the equation.
\begin{quote}\begin{description}
\sphinxlineitem{Parameters}\begin{itemize}
\item {} 
\sphinxAtStartPar
\sphinxstyleliteralstrong{\sphinxupquote{a}} \textendash{} Coefficient of x\textasciicircum{}2.

\item {} 
\sphinxAtStartPar
\sphinxstyleliteralstrong{\sphinxupquote{b}} \textendash{} Coefficient of x.

\item {} 
\sphinxAtStartPar
\sphinxstyleliteralstrong{\sphinxupquote{c}} \textendash{} Constant term.

\end{itemize}

\sphinxlineitem{Returns}
\sphinxAtStartPar
A tuple containing the two solutions for x.

\sphinxlineitem{Return type}
\sphinxAtStartPar
tuple of floats

\end{description}\end{quote}

\end{fulllineitems}

\index{sum\_of\_squares() (in module newtest)@\spxentry{sum\_of\_squares()}\spxextra{in module newtest}}

\begin{fulllineitems}
\phantomsection\label{\detokenize{newtest:newtest.sum_of_squares}}
\pysigstartsignatures
\pysiglinewithargsret{\sphinxcode{\sphinxupquote{newtest.}}\sphinxbfcode{\sphinxupquote{sum\_of\_squares}}}{\sphinxparam{\DUrole{n}{n}}}{}
\pysigstopsignatures
\sphinxAtStartPar
Computes the sum of squares of the first \(n\) integers:
\begin{equation*}
\begin{split}S = 1^2 + 2^2 + \cdots + n^2 = \frac{n(n+1)(2n+1)}{6}\end{split}
\end{equation*}
\sphinxAtStartPar
where:
\sphinxhyphen{} \(S\) is the sum of squares.
\sphinxhyphen{} \(n\) is the number of terms.
\begin{quote}\begin{description}
\sphinxlineitem{Parameters}
\sphinxAtStartPar
\sphinxstyleliteralstrong{\sphinxupquote{n}} \textendash{} The number of terms to sum.

\sphinxlineitem{Returns}
\sphinxAtStartPar
The sum of squares.

\sphinxlineitem{Return type}
\sphinxAtStartPar
float

\end{description}\end{quote}

\end{fulllineitems}


\sphinxstepscope


\section{pythagorus module}
\label{\detokenize{pythagorus:module-pythagorus}}\label{\detokenize{pythagorus:pythagorus-module}}\label{\detokenize{pythagorus::doc}}\index{module@\spxentry{module}!pythagorus@\spxentry{pythagorus}}\index{pythagorus@\spxentry{pythagorus}!module@\spxentry{module}}\index{area\_of\_circle() (in module pythagorus)@\spxentry{area\_of\_circle()}\spxextra{in module pythagorus}}

\begin{fulllineitems}
\phantomsection\label{\detokenize{pythagorus:pythagorus.area_of_circle}}
\pysigstartsignatures
\pysiglinewithargsret{\sphinxcode{\sphinxupquote{pythagorus.}}\sphinxbfcode{\sphinxupquote{area\_of\_circle}}}{\sphinxparam{\DUrole{n}{radius}}}{}
\pysigstopsignatures
\sphinxAtStartPar
Compute the area of a circle.

\sphinxAtStartPar
The formula is given by:
\begin{equation*}
\begin{split}A = \pi r^2\end{split}
\end{equation*}
\sphinxAtStartPar
where:
\sphinxhyphen{} \(A\) is the area,
\sphinxhyphen{} \(r\) is the radius.
\begin{quote}\begin{description}
\sphinxlineitem{Parameters}
\sphinxAtStartPar
\sphinxstyleliteralstrong{\sphinxupquote{radius}} (\sphinxstyleliteralemphasis{\sphinxupquote{float}}) \textendash{} The radius of the circle.

\sphinxlineitem{Returns}
\sphinxAtStartPar
The computed area.

\sphinxlineitem{Return type}
\sphinxAtStartPar
float

\end{description}\end{quote}

\end{fulllineitems}

\index{energy() (in module pythagorus)@\spxentry{energy()}\spxextra{in module pythagorus}}

\begin{fulllineitems}
\phantomsection\label{\detokenize{pythagorus:pythagorus.energy}}
\pysigstartsignatures
\pysiglinewithargsret{\sphinxcode{\sphinxupquote{pythagorus.}}\sphinxbfcode{\sphinxupquote{energy}}}{\sphinxparam{\DUrole{n}{mass}}\sphinxparamcomma \sphinxparam{\DUrole{n}{c}}}{}
\pysigstopsignatures
\sphinxAtStartPar
Computes energy using Einstein’s equation:
\begin{equation*}
\begin{split}E=mc^2\end{split}
\end{equation*}
\sphinxAtStartPar
where:
\sphinxhyphen{} \(E\) is energy,
\sphinxhyphen{} \(m\) is mass,
\sphinxhyphen{} \(c\)    is the speed of light.
\begin{quote}\begin{description}
\sphinxlineitem{Parameters}\begin{itemize}
\item {} 
\sphinxAtStartPar
\sphinxstyleliteralstrong{\sphinxupquote{mass}} \textendash{} The mass of the object.

\item {} 
\sphinxAtStartPar
\sphinxstyleliteralstrong{\sphinxupquote{c}} \textendash{} The speed of light.

\end{itemize}

\sphinxlineitem{Returns}
\sphinxAtStartPar
The computed energy.

\end{description}\end{quote}

\end{fulllineitems}

\index{force() (in module pythagorus)@\spxentry{force()}\spxextra{in module pythagorus}}

\begin{fulllineitems}
\phantomsection\label{\detokenize{pythagorus:pythagorus.force}}
\pysigstartsignatures
\pysiglinewithargsret{\sphinxcode{\sphinxupquote{pythagorus.}}\sphinxbfcode{\sphinxupquote{force}}}{\sphinxparam{\DUrole{n}{mass}}\sphinxparamcomma \sphinxparam{\DUrole{n}{acceleration}}}{}
\pysigstopsignatures
\sphinxAtStartPar
Computes force using Newton’s Second Law:
.. math::
F = m cdot a

\sphinxAtStartPar
where:
\sphinxhyphen{} \(F\) is the force,
\sphinxhyphen{} \(m\) is the mass,
\sphinxhyphen{} \(a\) is the acceleration.
\begin{quote}\begin{description}
\sphinxlineitem{Parameters}\begin{itemize}
\item {} 
\sphinxAtStartPar
\sphinxstyleliteralstrong{\sphinxupquote{mass}} \textendash{} The object’s mass.

\item {} 
\sphinxAtStartPar
\sphinxstyleliteralstrong{\sphinxupquote{acceleration}} \textendash{} The object’s acceleration.

\end{itemize}

\sphinxlineitem{Returns}
\sphinxAtStartPar
The computed force.

\end{description}\end{quote}

\end{fulllineitems}

\index{kinetic\_energy() (in module pythagorus)@\spxentry{kinetic\_energy()}\spxextra{in module pythagorus}}

\begin{fulllineitems}
\phantomsection\label{\detokenize{pythagorus:pythagorus.kinetic_energy}}
\pysigstartsignatures
\pysiglinewithargsret{\sphinxcode{\sphinxupquote{pythagorus.}}\sphinxbfcode{\sphinxupquote{kinetic\_energy}}}{\sphinxparam{\DUrole{n}{mass}}\sphinxparamcomma \sphinxparam{\DUrole{n}{velocity}}}{}
\pysigstopsignatures
\sphinxAtStartPar
Computes kinetic energy using the formula:
\begin{equation*}
\begin{split}KE = \frac{1}{2} m v^2\end{split}
\end{equation*}\begin{quote}\begin{description}
\sphinxlineitem{Parameters}\begin{itemize}
\item {} 
\sphinxAtStartPar
\sphinxstyleliteralstrong{\sphinxupquote{mass}} (\sphinxstyleliteralemphasis{\sphinxupquote{float}}) \textendash{} The object’s mass (\(m\)).

\item {} 
\sphinxAtStartPar
\sphinxstyleliteralstrong{\sphinxupquote{velocity}} (\sphinxstyleliteralemphasis{\sphinxupquote{float}}) \textendash{} The object’s velocity (\(v\)).

\end{itemize}

\sphinxlineitem{Returns}
\sphinxAtStartPar
The kinetic energy (\(KE\)).

\sphinxlineitem{Return type}
\sphinxAtStartPar
float

\end{description}\end{quote}

\end{fulllineitems}

\index{pythagoras() (in module pythagorus)@\spxentry{pythagoras()}\spxextra{in module pythagorus}}

\begin{fulllineitems}
\phantomsection\label{\detokenize{pythagorus:pythagorus.pythagoras}}
\pysigstartsignatures
\pysiglinewithargsret{\sphinxcode{\sphinxupquote{pythagorus.}}\sphinxbfcode{\sphinxupquote{pythagoras}}}{\sphinxparam{\DUrole{n}{a}}\sphinxparamcomma \sphinxparam{\DUrole{n}{b}}}{}
\pysigstopsignatures\begin{description}
\sphinxlineitem{Computes the hypotenuse using the Pythagorean theorem:}\begin{align*}\!\begin{aligned}
c^2=a^2+b^2\\
(a+b)^2=a^2+2ab+b^2\\
\end{aligned}\end{align*}
\end{description}
\begin{quote}\begin{description}
\sphinxlineitem{Parameters}\begin{itemize}
\item {} 
\sphinxAtStartPar
\sphinxstyleliteralstrong{\sphinxupquote{a}} \textendash{} Side A length.

\item {} 
\sphinxAtStartPar
\sphinxstyleliteralstrong{\sphinxupquote{b}} \textendash{} Side B length.

\end{itemize}

\sphinxlineitem{Returns}
\sphinxAtStartPar
Hypotenuse length.

\end{description}\end{quote}

\end{fulllineitems}

\index{velocity() (in module pythagorus)@\spxentry{velocity()}\spxextra{in module pythagorus}}

\begin{fulllineitems}
\phantomsection\label{\detokenize{pythagorus:pythagorus.velocity}}
\pysigstartsignatures
\pysiglinewithargsret{\sphinxcode{\sphinxupquote{pythagorus.}}\sphinxbfcode{\sphinxupquote{velocity}}}{\sphinxparam{\DUrole{n}{distance}}\sphinxparamcomma \sphinxparam{\DUrole{n}{time}}}{}
\pysigstopsignatures
\sphinxAtStartPar
Computes velocity using the equation:
\begin{equation*}
\begin{split}v =\frac{d}{t}\end{split}
\end{equation*}\begin{quote}\begin{description}
\sphinxlineitem{Parameters}\begin{itemize}
\item {} 
\sphinxAtStartPar
\sphinxstyleliteralstrong{\sphinxupquote{distance}} (\sphinxstyleliteralemphasis{\sphinxupquote{float}}) \textendash{} Distance traveled (\(d\)).

\item {} 
\sphinxAtStartPar
\sphinxstyleliteralstrong{\sphinxupquote{time}} (\sphinxstyleliteralemphasis{\sphinxupquote{float}}) \textendash{} Time taken (\(t\)).

\end{itemize}

\sphinxlineitem{Returns}
\sphinxAtStartPar
Computed velocity (\(v\)).

\sphinxlineitem{Return type}
\sphinxAtStartPar
float

\end{description}\end{quote}

\end{fulllineitems}


\sphinxstepscope


\section{subtract module}
\label{\detokenize{subtract:module-subtract}}\label{\detokenize{subtract:subtract-module}}\label{\detokenize{subtract::doc}}\index{module@\spxentry{module}!subtract@\spxentry{subtract}}\index{subtract@\spxentry{subtract}!module@\spxentry{module}}

\subsection{Subtraction Module}
\label{\detokenize{subtract:subtraction-module}}\begin{equation*}
\begin{split}a - b\end{split}
\end{equation*}
\sphinxAtStartPar
This module provides a function to subtract two numbers.
\index{subtract() (in module subtract)@\spxentry{subtract()}\spxextra{in module subtract}}

\begin{fulllineitems}
\phantomsection\label{\detokenize{subtract:subtract.subtract}}
\pysigstartsignatures
\pysiglinewithargsret{\sphinxcode{\sphinxupquote{subtract.}}\sphinxbfcode{\sphinxupquote{subtract}}}{\sphinxparam{\DUrole{n}{a}}\sphinxparamcomma \sphinxparam{\DUrole{n}{b}}}{}
\pysigstopsignatures
\sphinxAtStartPar
Returns the difference of a and b.

\end{fulllineitems}



\renewcommand{\indexname}{Python Module Index}
\begin{sphinxtheindex}
\let\bigletter\sphinxstyleindexlettergroup
\bigletter{a}
\item\relax\sphinxstyleindexentry{add}\sphinxstyleindexpageref{add:\detokenize{module-add}}
\indexspace
\bigletter{n}
\item\relax\sphinxstyleindexentry{newtest}\sphinxstyleindexpageref{newtest:\detokenize{module-newtest}}
\indexspace
\bigletter{p}
\item\relax\sphinxstyleindexentry{pythagorus}\sphinxstyleindexpageref{pythagorus:\detokenize{module-pythagorus}}
\indexspace
\bigletter{s}
\item\relax\sphinxstyleindexentry{subtract}\sphinxstyleindexpageref{subtract:\detokenize{module-subtract}}
\end{sphinxtheindex}

\renewcommand{\indexname}{Index}

\end{document}